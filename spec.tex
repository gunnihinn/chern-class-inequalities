\documentclass[11pt,a4paper]{amsart}

\usepackage[english]{babel}
\usepackage[T1]{fontenc}
\usepackage[utf8]{inputenc}
\usepackage{lmodern}
\usepackage{microtype}
\usepackage{amsmath,amsfonts,amssymb}
\usepackage[pdftex]{hyperref}
\usepackage{setspace}
\usepackage[all]{xy}
\usepackage{graphicx}
\usepackage{multirow}

\setstretch{1.1}

\title{Speculation on higher\\Chern class inequalities}

\def\^#1{^{[#1]}}
\def\bw#1{\bigwedge{}^{\mkern-5mu #1}\mkern2mu}
\def\I{\mathbf{I}}
\def\la{\langle}
\def\ra{\rangle}
\DeclareMathOperator{\tr}{tr}
\DeclareMathOperator{\id}{id}
\DeclareMathOperator{\Ric}{Ric}
\DeclareMathOperator{\End}{End}
\DeclareMathOperator{\Hom}{Hom}
\def\PP{\mathbb{P}}
\def\curv{\frac{i}{2\pi} \Theta}

\newtheorem{theo}{Theorem}[section]
\newtheorem{prop}[theo]{Proposition}
\newtheorem{coro}[theo]{Corollary}
\theoremstyle{definition}
\newtheorem{exam}[theo]{Example}
\theoremstyle{remark}
\newtheorem{rema}[theo]{Remark}
\numberwithin{equation}{section}

\begin{document}


\maketitle 


Let 
$$
f(z) = J_0(2\sqrt z) = \sum_{m\geq0} (-1)^m \frac{1}{m!^2} z^m.
$$
This is a Bessel function of the first kind and if we set $z = xy$ and look at its reciproque we find
$$
\frac{1}{f(xy)} = \sum_{l\geq 0} (-1)^l b_l x\^l y\^l.
$$
I want to find an interpretation of the norm formula that says something like
$$
\la u,v \ra \omega\^n
= 
\Bigl[
\frac{1}{f(xy)}(\Lambda,\Lambda)(u,\overline{\I v})
\wedge \exp(\omega)
\Bigr]\^n,
$$
where the $[n]$ exponent means projection to the $(n,n)$--th form. I would also be very happy if I could interpret the lower-degree terms on the right-hand side and get a formula like
$$
\alpha(u,v)
= 
\frac{1}{f(xy)}(\Lambda,\Lambda)(u,\overline{\I v})
\wedge \exp(\omega),
$$
where $\alpha$ is some sesquilinear operator on the exterior algebra.

Now, would there be any benefits to this, especially w/r/t our hope of calculating Chern numbers, or would this just make me feel good?

I wonder if the combinations of Chern classes that turn up in our hypothetical formulas come from Segre classes? Naively, a $(c,1/f)$ pair on one side corresponds to a $(1/c,f)$ pair on the other.


\end{document}