\documentclass[11pt,a4paper]{amsart}


\usepackage[english]{babel}
\usepackage[T1]{fontenc}
\usepackage[utf8]{inputenc}
\usepackage{lmodern}
\usepackage{microtype}
\usepackage{amsmath,amsfonts,amssymb}
\usepackage[pdftex]{hyperref}
\usepackage{setspace}
\usepackage[all]{xy}
\usepackage{graphicx}
\usepackage{multirow}

\setstretch{1.1}

\title{Higher Chern class inequalities}

\def\^#1{^{[#1]}}
\def\bw#1{\bigwedge{}^{\mkern-5mu #1}\mkern2mu}
\def\I{\mathbf{I}}
\def\la{\langle}
\def\ra{\rangle}
\def\Im{\operatorname{Im}}
\DeclareMathOperator{\tr}{tr}
\DeclareMathOperator{\id}{id}
\DeclareMathOperator{\Ric}{Ric}
\DeclareMathOperator{\End}{End}
\DeclareMathOperator{\Hom}{Hom}
\def\ZZ{\mathbb{Z}}
\def\CC{\mathbb{C}}
\def\PP{\mathbb{P}}
\def\RR{\mathcal{R}}
\def\OO{\mathcal{O}}
\def\curv{\frac{i}{2\pi} \Theta}

\newtheorem{theo}{Theorem}[section]
\newtheorem{prop}[theo]{Proposition}
\newtheorem{coro}[theo]{Corollary}
\theoremstyle{definition}
\newtheorem{exam}[theo]{Example}
\theoremstyle{remark}
\newtheorem{rema}[theo]{Remark}
\numberwithin{equation}{section}

\makeatletter
\def\eqalign#1{%
 \null\,\vcenter{\openup\jot\m@th
  \ialign{\strut\hfil$\displaystyle{##}$&$\displaystyle{{}##}$\hfil
      &&\hfil$\displaystyle{##}$&$\displaystyle{{}##}$\hfil\crcr#1\crcr}}\,}
\makeatother

\begin{document}


\maketitle 



\section*{Motivation}

Let $(X,\omega)$ be a Hermitian manifold of dimension $n$ and $(E,h) \to X$
a Hermitian vector bundle of rank $r$ over $X$. The Kobayashi--L\"{u}bke
inequality says that if $(E,h)$ is Hermite--Einstein, then we have
$$
\bigl((r-1)c_1(E)^2 - 2rc_2(E)\bigr) \wedge \omega^{n-2} \leq 0
$$
pointwise on $X$, where $c_k(E)$ is the $k$--th Chern form of $E$ defined by the curvature form of $h$.

The proof of this inequality is neither clear nor well motivated, at
least from the viewpoint of complex differential geometry. The quantity
that the products of Chern classes are estimated against comes from the
curvature form of the vector bundle, but in all proofs the writers seem
to know in advance what they want to prove and proceed by brutally
calculating all things in sight in local coordinates. This is probably a
product of the historical evolution of this inequality, which first rose as
an inequality between Chern classes on K\"ahler--Einstein manifolds, was
then proved for semistable vector bundles over projective surfaces and
finally extended to Hermite--Einstein vector bundles; the rough
chronological order of its evolution is
\cite{Chen-Ogiue,Miyaoka,Bogomolov,Gieseker,Lubke} -- see also
\cite{Kobayashi,LubkeStab,Siu} for the link between Hermite--Einstein
and semistable vector bundles.

I want to find a more natural proof of this inequality. Or rather, I want
to find the natural home of this inequality, bring out the quantities
that are being calculated and estimated and write them in coordinate-free
ways. I hope this will make the inequality clearer, though its proof will
quite likely be the same as the ones in \cite{Chen-Ogiue} or \cite{Lubke}
if we write it in local coordinates.

I also hope this will let us find similar inequalities for higher Chern
classes. In short, I hope the Kobayashi--L\"{u}bke inequality arises while
attempting to calculate the norm of the curvature tensor of $(E,h)$. One can
then hope to find inequalities between higher Chern classes by calculating
the norms of exterior powers of the curvature tensor. This may be related
to one of Yau's problems \cite[Problem~45]{Yau}: ``Let $M$ be a compact
K\"{a}hler--Einstein manifold of complex dimension $n$ with negative scalar
curvature. Yau~\cite{Yau2} proved then that
$$
(-1)^n \frac{2(n+1)}{n} c_1^{n-2} c_2 \geq (-1)^n c_1^n.
$$
Are there any other inequalities of this sort among the Chern numbers
of~$M$?''%

To do any of this we first need to express the square of the norm of a
$(k,k)$--form on a vector space as a linear combination of squares of its
various traces against the metric form. Such an expression is well-known
for the square of the norm of a $(1,1)$--form but do not seem to have been
worked out for the norms of higher-degree forms.




\section{Inner products of exterior forms}



Let $V$ be a complex vector space of dimension $n$ and $\omega$ a Hermitian
inner product on $V$. The Hodge star operator of $\omega$ is $*$,
the Lefschetz operator is $L$ and its adjoint is $\Lambda$. We write
$\I = \sum_{p,q} i^{p-q} \pi_{p,q}$, where $\pi_{p,q} : \bw{*} V^* \to
\bw{p,q} V^*$ is the orthogonal projection.

Recall that a $k$--form $u$ on $V$ is \emph{primitive} if $\Lambda u = 0$.
This is equivalent to $L^{n-k-1}u = 0$. Any $k$--form $u$ on $V$ admits a
\emph{primitive decomposition} $u = \sum L^{k-j} u_j$, which is an
orthogonal decomposition of $u$ where each form $u_j$ is a primitive
$(k-2j)$--form.

If $A$ is an element of an algebra then we define $A\^k = A^k / k!$ for $k
\geq 0$. This entails that
$$
A\^j \cdot A\^k = \tbinom{j+k}{j} A\^{j+k}.
$$
We will use this convention for the element $\omega$ of the exterior
algebra $\bw{*} V^*$ and the operators $L$ and $\Lambda$ on that
algebra.



Consider the sequence of integers defined recursively by $b_0 = 1$ and
\begin{equation}
\label{bezel}
\sum_{l=0}^{p} (-1)^{l} \tbinom{p}{l}^2 b_l = 0
\end{equation}
for $p \geq 1$. This is sequence number A000275 in the \textsl{On-line
encyclopedia of integer sequences}~\cite{OEIS}; see also \cite{Carlitz}
and \cite{Riordan} for perhaps unrelated information about the sequence.
Its first few values are: 
$$
1 \quad 1 \quad 3 \quad 19 \quad 211 \quad 3651 \quad
90.921 \quad 3.081.513 \quad 136.407.699 \quad 7.642.177.651
$$
Our objective in this section is to prove the following theorem.


\begin{theo}
    \label{thm:norm}
Let $u$ be a $k$--form on an $n$--dimensional complex vector
space $V$. Then
$$
    (-1)^{k(k+1)/2}
    \lvert u \rvert^2 \omega\^n
    = \sum_{l=0}^n (-1)^{l} b_l \, 
    (\Lambda\^l u \wedge \Lambda\^l \overline{\I u})
    \wedge \omega\^{n-k+2l}.
$$
\end{theo}

We will actually prove this result for a $(p,q)$--form $u$ with $p \leq q$
and $p+q = k$. By conjugation the restriction $p \leq q$ is irrelevant and
it is just a matter of basic combinatorics and degree reasoning to see
that if the result holds for $(p,q)$--forms with $p + q = k$ then it also
holds for a $k$--form, i.e., that the right-hand side respects the
orthogonal decomposition of $\bw{k}V^*$.  Proving this result demands a
certain amount of preparation, all of which rests on the following formula.


\begin{prop}[{{\cite[Proposition~1.67]{Huy}}}]
Let $u$ be a primitive $(p,q)$--form on $V$. Set $k = p+q$. Then
$$
    * L\^{j} u = (-1)^{k(k+1)/2} L\^{n-j-k} \I u.
$$
\end{prop}


\begin{exam}
If $u$ is a primitive $(p,q)$--form, then
\begin{align*}
\lvert L\^{j} u \rvert^2 \omega\^{n}
= L\^{j} u \wedge * L\^{j} \overline{\I u}
&= L\^{j} u \wedge (-1)^{k(k+1)/2} L\^{n-j-k} \overline{\I u} \\
&= u \wedge \tbinom{n-k}{j} (-1)^{k(k+1)/2} L\^{n-k} \overline{\I u} \\
&= \tbinom{n-k}{j} u \wedge * \overline{\I u}
= \tbinom{n-k}{j} \lvert u \rvert^2 \omega\^{n},
\end{align*}
whose moral is that $\lvert L\^{j} u \rvert^2 = \tbinom{n-k}{j} \lvert u
\rvert^2$. Taking $u = 1$ gives $|\omega\^{j}|^2 = \binom{n}{j}$.
\end{exam}


If $u$ is a $(p,q)$--form with $p \leq q$, then we write $u =
\sum_{j=0}^p L\^{p-j} u_j$ for its primitive decomposition, where each
$u_j$ is a primitive $(j,j+q-p)$--form. This decomposition is
orthogonal, so
$$
\lvert u \rvert^2
= \sum_{j=0}^p \lvert L\^{p-j} u_j \rvert^2
= \sum_{j=0}^p \tbinom{n-2j-q+p}{p-j} \lvert u_j \rvert^2.
$$


\begin{prop}
  \label{16}
  Let $u = \sum_{j=0}^p u_j \wedge \omega^{p-j}$ be the primitive
decomposition of a $(p,q)$--form $u$, where $p \leq q$ and $u_j$ is a
primitive $(j,j+q-p)$--form. Then
$$
u \wedge \overline{\I u} \wedge \omega^{n-p-q}
= \sum_{j=0}^p u_j \wedge \overline{\I u_j} \wedge \omega^{n-2j-q+p}.
$$
\end{prop}

\begin{proof}
  We induct on $p$, the result being clear for $p = 0$ or $q = 0$. If
$u$ is a $(p+1,q+1)$--form we have $u = \omega \wedge v_p + u_{p+1}$,
where $v_p$ is a $(p,q)$--form whose primitive decomposition is
evident. Then
$$
u \wedge \overline{\I u} = \omega^2 \wedge v_p \wedge \overline{\I v_p}
+ \omega \wedge v_p \wedge \overline{\I u_{p+1}}
+ \omega \wedge u_{p+1} \wedge \overline{\I v_p}
+ u_{p+1} \wedge \overline{\I u_{p+1}},
$$
so
$$
\displaylines{
u \wedge \overline{\I u} \wedge \omega^{n-p-q-2}
= v_p \wedge \overline{\I v_p} \wedge \omega^{n-p-q}
+ v_p \wedge \overline{\I u_{p+1}} \wedge \omega^{n-p-q-1}
\hfill\cr\hfill
{}+ u_{p+1} \wedge \overline{\I v_p} \wedge \omega^{n-p-q-1}
+ u_{p+1} \wedge \overline{\I u_{p+1}} \wedge \omega^{n-p-q-2}.
}
$$
The two middle terms are zero because $u_{p+1}$ is primitive, so $u_{p+1}
\wedge \omega^{n-p-q-1} = 0$ and $v_p \wedge \overline{\I v_p} \wedge
\omega^{n-p-q}$ is of the announced form by induction.
\end{proof}


\begin{rema}
  Here the reader may wonder what happens for a $(p,q)$--form with
$p+q > n$. The above proposition then just says that $0 = 0$, which is
not very profound. We should however stress that $p + q > n$ never
poses a problem in what follows: That $L^k : \bw{n-k} V^* \to \bw{n+k}
V^*$ is an isomorphism entails that there are no primitive
$(p,q)$--forms with $p + q > n$.
\end{rema}



\begin{prop}
    \label{prop:morphism}
Let $u = \sum_{j=0}^p u_j \wedge \omega\^{p-j}$ be the primitive
decomposition of a $(p,q)$--form $u$, where $p \leq q$ and each $u_j$ is a
primitive $(j,j+q-p)$--form. Then
\begin{gather}
  \label{18:1}
  \Lambda\^l u 
  = \sum_{j=0}^{p-l} \tbinom{n-j-q+l}{l} L\^{p-j-l} u_j , 
  \\
  L\^{n-p-q+2l}( \Lambda\^l u \wedge \Lambda\^l \overline{\I u})
  \kern 193pt
  \\
  \label{18:2}
  \kern 25pt
  = \sum_{j=0}^{p-l} 
  (-1)^j
  (-1)^{(q-p)(q-p+1)/2}
  \tbinom{p-j}{l}
  \tbinom{n-j-q+l}{p-l-j}
  \tbinom{n-j-q+l}{l}
  \lvert L\^{p-j} u_j \rvert^2
  \wedge \omega\^ n
  \notag
\end{gather}
and the decomposition of $\Lambda\^l u$ in \eqref{18:1} is primitive.
\end{prop}

\begin{proof}
  By linearity it is enough to prove \eqref{18:1} for a $(p,q)$--form
$u = L\^{p-j} u_j$, where $u_j$ is a primitive $(j,j+q-p)$--form. Let
$v$ be a form of degree $(n-p+l,n-q+l)$ and set $k = 2j+q-p$. Then
  \begin{align*}
    \la v, \Lambda\^l (L\^{p-j} u_j ) \ra \omega\^ n
    &= \la L\^l v, L\^{p-j} u_j \ra L\^ n \\
    &= L\^l v \wedge i^{q-p}(-1)^{\binom{k+1}{2}} L\^{n-j-q} \overline u_j  \\
    &= v \wedge i^{q-p}(-1)^{\binom{k+1}{2}}
    \tbinom{n-j-q+l}{l} L\^{n-j-q+l} \overline u_j  \\
    &= v \wedge * \bigl( \tbinom{n-j-q+l}{l} L\^{p-j-l} \overline u_j \bigr) \\
    &= \la v, \tbinom{n-j-q+l}{l} L\^{p-j-l} u_j \ra \omega\^ n.
  \end{align*}
Since the equality holds for all $v$, \eqref{18:1} is proved for forms
of type $L\^{p-j} u_j$, where $u_j$ is primitive. Note that the
decomposition of $\Lambda\^l (L\^{p-j} u_j)$ is again primitive, so
the same will hold for an arbitrary form $u$.

For \eqref{18:2} we first 
apply Proposition~\ref{16} to our $(p,q)$--form
$u = \sum_j L\^{p-j} u_j = \sum_j (u_j/(p-j)!) \wedge \omega^{p-j}$.
That gives
\begin{align*}
L\^{n-p-q} (u \wedge \overline{\I u})
&= \sum_{j=0}^p \frac{1}{(n-p-q)!}
\Bigl( \frac{u_j\wedge \overline{\I u_j}}{(p-j)!^2} \Bigr) 
\wedge \omega^{n-2j-q+p}\\
&= \sum_{j=0}^p \frac{1}{(n-p-q)!}\frac{(n-2j-q+p)!}{(p-j)!^2} 
u_j \wedge \overline{\I u_j} \wedge \omega\^{n-2j-q+p}\\
&= \sum_{j=0}^p \tbinom{n-j-q}{p-j} \tbinom{n-2j-q+p}{p-j}
u_j \wedge \overline{\I u_j} \wedge \omega\^{n-2j-q+p}
\\
&= \sum_{j=0}^p 
(-1)^{\binom{2j+q-p+1}{2}} 
\tbinom{n-j-q}{p-j} \tbinom{n-2j-q+p}{p-j}
\lvert u_j \rvert^2 \omega\^{n}
\\
&= \sum_{j=0}^p (-1)^j(-1)^{(q-p)(q-p+1)/2}
\tbinom{n-j-q}{p-j} \tbinom{n-2j-q+p}{p-j}
\lvert u_j \rvert^2 \omega\^{n}.
\end{align*}
To get the general result, we apply this to the $(p-l,q-l)$--form
$\Lambda\^l u$. That gives (by letting $p \mapsto p - l$, $q \mapsto q
- l$)
$$
\displaylines{
  L\^{n-p-q+l} (\Lambda\^l u \wedge \Lambda\^l \overline{\I u})
  \hfill\cr\hfill
  \jot=0pt
  \eqalign{
  &= \sum_{j=0}^{p-l} 
  (-1)^{\binom{2j+q-p+1}{2}} 
  \tbinom{n-2j-q+p}{p-l-j}
  \tbinom{n-j-q+l}{p-l-j}
  \tbinom{n-j-q+l}{l}^2
  \lvert u_j \rvert^2
  \wedge \omega\^ n
  \cr
  &= \sum_{j=0}^{p-l} 
  (-1)^{\binom{2j+q-p+1}{2}} 
  \frac{
  \tbinom{n-2j-q+p}{p-l-j}
  \tbinom{n-j-q+l}{p-l-j}
  \tbinom{n-j-q+l}{l}^2
  }{\tbinom{n-2j-q+p}{p-j}}
  \lvert L\^{p-j} u_j \rvert^2
  \wedge \omega\^ n.
  }\!
}
$$
Once we remark that
$$
\frac{\binom{n-2j-q+p}{p-j-l} \binom{n-j-q+l}{l}}{\binom{n-2j-q+p}{p-j}}
= \tbinom{p-j}{l}
$$
the proof is finished.
\end{proof}


This last result is the key to proving what we want. It tells us how to
write the square of the norm of a $(p,q)$--form $u$ with $p \leq q$ as
a linear combination of traces of the form: Define two vector spaces
\begin{align*}
X &= \operatorname{Span}(L\^{n-p-q+2l)} 
(\Lambda\^l u \wedge \Lambda\^l \overline{\I u}) \mid l = 0,\ldots,p),
\\
Y &= \operatorname{Span}(|L\^{p-j}u_j|^2 \omega\^{n} \mid j=0,\ldots,p).
\end{align*}
(Consider these as formal symbols if the possibility and implications of
one of them being zero for a given form worries you.) Equation~\eqref{18:2}
defines a linear morphism $A : X \to Y$; a morphism that only depends
on the dimension of $V$ and the degree $p$, but is otherwise independent of
the form $u$. Since $|u|^2\omega\^n = \sum |L\^{p-j}u_j| \omega\^n$,  
the coefficients of the linear combination we seek are the coordinates of
the vector $A^{-1}(1,\ldots,1)$. This observation shows that coefficients
like the ones we seek exist, the task is now to show that they coincide
with our integer sequence.



\begin{proof}[Proof of Theorem~\ref{thm:norm}]
    If $u$ is a $(p,q)$--form with $p \leq q$ then we set $k = p+q$
and write
$$
\lvert u \rvert^2 \omega\^n
= \sum_{l=0}^n (-1)^{l + k(k+1)/2} b_l(p,n) \, 
(\Lambda\^l u \wedge \Lambda\^l \overline{\I u})
\wedge \omega\^{n-p-q+2l},
$$
where $b_l(p,n)$ is the coefficient whose existence is guaranteed by
Proposition~\ref{prop:morphism}.  We will prove that $b_l(p,n) = b_l$ by
induction on $p$. 

We first remark that $b_0(p,n) = b_0 = 1$ for all $p, q, n$, because
the norm of a primitive $(p,q)$--form $u$ is $\lvert u \rvert^2 \omega\^n =
(-1)^{k(k+1)/2} u \wedge \overline{\I u} \wedge \omega\^{n-p-q}$.

For the induction step, we assume that $b_l(p,n) = b_l$ for $l = 0, \ldots,
p-1$ and want to prove that $b_{p}(p,n) = b_{p}$. For this, first
recall that if $u = \sum_j L\^{p-j} u_j$ is the primitive decomposition of
a $(p,q)$--form with $p \leq q$, then 
$$
\lvert u \rvert^2 \omega\^n 
= \sum_j \lvert L\^{p-j} u_j \rvert^2.
$$ 
Let's record for immediate use that if $p \leq q$ then
$(-1)^{(q-p)(q-p+1)/2} = (-1)^p(-1)^{(p+q)(p+q+1)/2}$. Then we can also write
the above as
$$
\displaylines{
\lvert u \rvert^2 \omega\^n 
= \sum_{l=0}^n (-1)^{k(k+1)/2+l} b_l(p,n) \, 
(\Lambda\^l u \wedge \Lambda\^l \overline{\I u})
\wedge \omega\^{n-2(p-l)}
\hfill\cr
\phantom{\lvert u \rvert^2 \omega\^n}
{}= \sum_{l=0}^n (-1)^{k(k+1)/2+l} b_l(p,n) 
\sum_{j=0}^l 
(-1)^{j}
(-1)^{(q-p)(q-p+1)/2}
\hfill\cr\hfill{}\times
  \tbinom{p-j}{l}
  \tbinom{n-j-q+l}{n-p-q+2l}
  \tbinom{n-j-q+l}{l}
  \lvert L\^{p-j} u_j \rvert^2
  \wedge \omega\^ n
\cr
\phantom{\lvert u \rvert^2 \omega\^n}
{}= \sum_{l=0}^n (-1)^{l+p} b_l(p,n) \, 
\sum_{j=0}^l 
(-1)^j
  \tbinom{p-j}{l}
  \tbinom{n-j-q+l}{n-p-q+2l}
  \tbinom{n-j-q+l}{l}
  \lvert L\^{p-j} u_j \rvert^2
  \wedge \omega\^ n.
  \hfill
}
$$
By comparing the coefficients of $\lvert L\^p u_0 \rvert^2$ in these
two expressions we find that
\begin{align*}
1 &= 
\sum_{l=0}^p (-1)^{l+p} b_l(p,n) 
\, 
\tbinom{p}{l}
\tbinom{n-q+l}{n-p-q+2l}
\tbinom{n-q+l}{l}
\\
&= 
\tbinom{n}{p} b_p(p,n)
+ \sum_{l=0}^{p-1} 
(-1)^{l+p}
b_l \, 
\tbinom{p}{l}
\tbinom{n-q+l}{p-l}
\tbinom{n-q+l}{l}
\end{align*}
for all $n \geq p+q$. The binomial coefficient $\binom{n}{k}$ is a
polynomial of degree $k$ in $n$ whose leading term is
$1/k!$. Comparing the top-degree coefficients of $n$ in the above
equation we find that
$$
0 =
\frac{1}{p!} b_p(p,n)
+ \sum_{l=0}^{p-1} (-1)^{l+p} b_l \, 
\tbinom{p}{l}
\frac{1}{l! (p-l)!}.
$$
Since this equation expresses $b_p(p,n)$ in terms of things that do not
depend on $n$, we conclude that $b_p$ doesn't depend on $n$ either.
The defining recurrance relation \eqref{bezel} for the integers $b_l$,
now shows that $b_p(p,n) = b_p$.

Finally, we remark that by conjugating the form $u$, exchanging $p$ and
$q$, and carefully calculating the resulting sign change, it is enough
to prove our formula for forms $u$ with $p \leq q$.
\end{proof}


Polarizing now gives the general inner product on $(p,q)$--forms:

\begin{coro}
If $u$ and $v$ are complex $(p,q)$--forms on $V$, then
$$
(-1)^{k(k+1)/2}
\la u, v \ra \, \omega\^n
= \sum_{l=0}^{n} 
(-1)^{l} b_l \, 
(\Lambda\^l u) \wedge (\Lambda\^l \overline{\I v}) \wedge \omega\^{n-p-q+2l}.
$$
\end{coro}


\begin{exam}
In addition to the well-known formula for the square of the norm of a
$(1,1)$--form, Theorem~\ref{thm:norm} gives these formulas for the norms
of higher-degree forms, that have not appeared before to the best of my
knowledge.

\smallskip
\noindent
(i)\quad
For a $(2,2)$--form $u$ on $V$ we have
$$
|u|^2 \omega\^{n}
= u^2 \wedge \omega\^{n-4}
- (\Lambda u)^2 \wedge \omega\^{n-2}
+ 3 (\Lambda\^{2} u)^2 \wedge \omega\^{n}.
$$
    
\smallskip
\noindent
(ii)\quad
For a $(3,3)$--form $u$ our formula gives
$$
|u|^2 \omega\^{n}
= 
- u^2 \wedge \omega\^{n-6}
+ (\Lambda u)^2 \wedge \omega\^{n-4}
- 3 (\Lambda\^{2} u)^2 \wedge \omega\^{n-2}
+ 19 (\Lambda\^{3} u)^2 \wedge \omega\^{n}.
$$

\smallskip
\noindent
(iii)\quad
For a $(4,4)$--form $u$ we get
$$
\displaylines{
|u|^2 \omega\^{n}
= 
 u^2 \wedge \omega\^{n-8}
- (\Lambda u)^2 \wedge \omega\^{n-6}
+ 3 (\Lambda\^{2} u)^2 \wedge \omega\^{n-4}
\hfill\cr\hfill
{}- 19 (\Lambda\^{3} u)^2 \wedge \omega\^{n-2}
+ 211 (\Lambda\^{4} u)^2 \wedge \omega\^{n}.
}
$$
\end{exam}


Our theorem allows us to express the scalar product of two forms as a
wedge product of forms derived from the original ones. Doing things the
other way around, or expressing a wedge product in terms of inner products
is also possible:


\begin{coro}
    If $u$ and $v$ are $(p,q)$--forms on $V$, then
    $$
    (-1)^{k(k+1)/2} u \wedge \overline{\I v} \wedge \omega\^{n-p-q}
    = \sum_{m = 0}^n (-1)^{m} 
    \la \Lambda\^m u , \Lambda\^m \overline{\I v} \ra
    \wedge \omega\^{n-p-q-2m}.
    $$
\end{coro}


\begin{proof}
We remark that as usual it is enough to prove our statement for
$(p,q)$--forms with $p \leq q$, so we assume this holds. Plugging
$\Lambda\^m u$ and $\Lambda\^m \I v$ into our formula gives
$$
\displaylines{
(-1)^{k_m(k_m+1)/2} 
\la \Lambda\^m u , \Lambda\^m \overline{\I v} \ra
\wedge \omega\^{n-p-q-2m}
\hfill\cr\hfill
{} = \sum_{l=0}^{n} 
(-1)^l b_l \, 
\tbinom{l+m}{l}^2
(\Lambda\^{l+m} u) \wedge (\Lambda\^{l+m} \overline{\I v}) 
\wedge \omega\^{n-p-q+2(l+m)},
}
$$
where we write $k_m = p+q-2m$. Remark that 
$$
(-1)^{k_m(k_m+1)/2} =
(-1)^{k(k+1)/2} (-1)^m.
$$  
If we sum both sides of the above equation for the scalar product over $m$
from $0$ to $n$ and then change the variable in the first sum from $m$ to
$\nu = l+m$ we get 
$$
\displaylines{
    (-1)^{k(k+1)/2} 
    \sum_{m=0}^n
    (-1)^{m} \la \Lambda\^m u , \Lambda\^m \overline{\I v} \ra
    \wedge \omega\^{n-p-q+2(l+m)}
    \hfill\cr\noalign{\vskip-3pt}\hfill
    \jot=0pt
    \eqalign{
    {} &= 
    \sum_{\nu=0}^{n} 
    \Bigl(
    \sum_{l=0}^{\nu}
    (-1)^l b_l \, 
    \tbinom{\nu}{l}^2
    \Bigr)
    (\Lambda\^{\nu} u) \wedge (\Lambda\^{\nu} \overline{\I v}) 
    \wedge \omega\^{n-p-q+2\nu)}
    \cr\noalign{\vskip3pt}{}
    &= u \wedge \overline{\I v} \wedge \omega\^{n-p-q},
}
}
$$
because $\sum_{l=0}^{\nu} (-1)^l b_l \, \tbinom{\nu}{l}^2 = 0$ for all $\nu \geq 1$ by definition.
\end{proof}



\section{Curvature tensors and the Kobayashi--L\"{u}bke inequality}


\subsection*{Linear algebraic preliminaries}

Let $V$ and $E$ be complex vector space of dimensions $n$ and $r$, equipped
with Hermitian metrics $\omega$ and $h$. Let $R$ be a curvature-type
tensor, or an element of $\bigwedge^{1,1} V^* \otimes \End E$
that is Hermitian. We view $R$ as a $(2,2)$--form on the space $E \oplus
V$, equipped with the Hermitian metric $\alpha = \omega + h$, where we
abuse notation and do not write $\alpha = p_V^*\omega + p_E^*h$ as we should.

Let $e$ be a form on $E$ and $v$ a form on $V$. By picking orthonormal
coordinates we quickly verify that
$$
*_\alpha(p_V^* v \wedge p_E^* e) 
= p_V^*(*_\omega v) \wedge p_E^*(*_h e).
$$
Since the exterior algebra of $V \oplus E$ is generated by elements of the
type $p_V^* v \wedge p_E^* e$ this lets us calculate with the Hodge star
operator on that space.

Let $\Lambda_\omega$ and $\Lambda_h$ be the adjoints of the Lefschetz
operators of $\omega$ and $h$, pulled back to $V \oplus E$. I claim these
operators commute; since they are basically the trace operators in disguise
this should follow from general things in Coffman's~\cite{Coffman}. 

We also note that Newton's binomial formula gives
$$
\alpha\^{l} = \sum_{k=0}^{l} \omega\^{k} \wedge h\^{l-k}
$$
and that many of those terms will be zero for $l$ big for degree
reasons. A similar formula expresses $\Lambda\^{l}_\alpha$ in terms of
$\Lambda\^{k}_\omega$ and $\Lambda\^{l-k}_h$.

Finally we set $k! c_k := \Lambda\^{k}_h (\bigwedge^k \! R)$ for $k = 0, \ldots,
r$.  The notation is so chosen because when $R$ is the curvature tensor of
an actual Hermitian metric on a vector bundle, the $c_k$ will be the Chern
forms defined by $R$. The $k!$ factor deserves an explanation:

The inner product $h$ is an isomorphism $h : E \to \overline E^*$. It
induces inner products on both $\End E$ and $\bigwedge^{1,1} \! E^*$ and a
morphism $h  \otimes \id_{E^*}: \End E \to \bigwedge^{1,1} \!
E^*$. The trace of an endomorphism of $E$ is just its scalar product
again the identity morphism.  Taking $k$--th exterior powers we get a
canonical morphism
$h^k \otimes \id_{\wedge^k E^*} : \End \bigwedge^k \! E \to
\bigwedge^{k,k} \! E^*$. This morphism is however not an isometry, but
$h\^k$ is; this can be seen by comparing the norms of $\id_{\wedge^k
E}$ and its image $h\^k$. We now want to find a morphism
$\bigwedge^{1,1} \! E \to \bigwedge^{k,k} \! E$ that makes the diagram
$$
\xymatrix@C+10pt{
    \End E \ar[r]^-{f \mapsto \wedge^k f} \ar[d]^{h \otimes \id_{E^*}} & 
    \End \bigwedge^{k} \! E \ar[d]^{h\^k \otimes \id_{\wedge^k E^*}} \\
    \bigwedge^{1,1} \! E \ar@{-->}[r] 
    & \bigwedge^{k,k} \! E
}
$$
commute. This morphism is clearly $u \mapsto u^k / k!$, whence the factor
of $k!$ above.



\subsection*{The norm of a curvature tensor}

Our first result here is on the norm of a curvature tensor of a vector
bundle. This equality is implicit in the literature on the
Kobayashi--L\"{u}bke inequality (compare with \cite{Chen-Ogiue},
\cite{Lubke} and \cite{Siu}); the inequality is actually a corollary of a
simple application of Cauchy--Schwarz to the equation for the norm of the
curvature tensor.

\begin{theo}
    Let $E \to X$ be a holomorphic vector bundle of rank $r$ over a complex
manifold $X$ of dimension $n$. Let $\omega$ and $h$ be Hermitian metrics on
$X$ and $E$, respectively. Let $\curv$ be the curvature form of $(E,h)$
and let $c_k$ be the Chern forms defined by the curvature form. Then
$$
\Bigl\lvert \curv \Bigr\rvert^2 \omega\^n
= (2 c_2 - c_1^2) \wedge \omega\^{n-2}
+ \Bigl\lvert \tr_\omega \curv \Bigr\rvert^2 \omega\^n
$$
at every point of $X$. If $(E,h)$ is Hermite--Einstein, then we also have
$$
0 \leq 
(2r c_2 - (r-1) c_1^2) \wedge \omega\^{n-2}
$$
pointwise on $X$ with equality if and only if $\curv = (\lambda/n) \id_E
\otimes\, \omega$, where $\lambda$ is the Hermite--Einstein constant of
$(E,h)$.
\end{theo}


\begin{proof}
   The announced result is local on $X$, so we pick a point $x \in X$ and
write $V = T_{X,x}$, abuse notation to write $E = E_{x}$ and write $R$
for the image of $\curv$ under the isometry $\bigwedge^{1,1} \! T_X
\otimes \End E \to \bigwedge^{2,2} (T_X \otimes E)^*$ defined by $h$. Then
$R$ is a $(2,2)$--form on $V \oplus E$.\footnote{Morally speaking, this
amounts to viewing $\curv$ as a form defined on the total space of the
vector bundle $E$ and equipping that space with a metric induced by the
ones on $X$ and $E$. Since the vector bundle is holomorphically locally
trivial, its tangent bundle splits holomorphically into pullbacks of $E$
and $T_X$ to the total space of $E$, so this viewpoint poses no global
problems.} We write $\alpha = \omega + h$ for the induced inner product on
$V \oplus E$, in slight abuse of notation.

The norm of $R$ as a $(2,2)$--form on $V \oplus E$ is
$$
|R|^2 \alpha\^{n+r}
= R^2 \wedge \alpha\^{n+r-4}
- (\Lambda_\alpha R)^2 \wedge \alpha\^{n+r-2}
+ 3 (\Lambda\^{2}_\alpha R)^2 \wedge \alpha\^{n+r}.
$$
We have
\begin{align*}
R^2 \wedge \alpha\^{n+r-4}
&= R^2 \wedge \biggl(\sum_{k=0}^4\omega\^{n-k} \wedge h\^{r+k-4}\biggr)
\\
&= R^2 \wedge \omega\^{n-2} \wedge h\^{r-2}
= 2 c_2 \wedge \omega\^{n-2} \wedge h\^{r},
\end{align*}
where the second equality holds for degree reasons. Similarly we get
\begin{align*}
(\Lambda_\alpha R)^2 \wedge \alpha\^{n+r-2}
&= (\Lambda_\omega R + \Lambda_h R)^2 \wedge \alpha\^{n+r-2}
\\
&= \bigl( (\Lambda_\omega R)^2 + 2 \Lambda_\omega R \wedge \Lambda_h R 
+ (\Lambda_h R)^2 \bigr) 
\wedge \alpha\^{n+r-2}
\\
&= (\Lambda_\omega R)^2 \wedge \omega\^{n} \wedge h\^{r-2} 
\\
&\qquad
+ 2 \Lambda_\omega R \wedge \Lambda_h R \wedge \omega\^{n-1} \wedge h\^{r-1}
+ c_1^2 \wedge \omega\^{n-2} \wedge h\^{r} 
\\
&= (\Lambda_\omega R)^2 \wedge \omega\^{n} \wedge h\^{r-2} 
\\
&\qquad
+ 2 (\tr_\omega c_1)^2 \omega\^{n} \wedge h\^{r}
+ c_1^2 \wedge \omega\^{n-2} \wedge h\^{r} 
\end{align*}
because 
\begin{align*}
\Lambda_\omega R \wedge \Lambda_h R \wedge \omega\^{n-1} \wedge h\^{r-1}
&= \Lambda_h \Lambda_\omega R \cdot \Lambda_\omega \Lambda_h R \;
\omega\^{n} \wedge h\^{r}
\\
&= (\tr_\omega c_1)^2 \omega\^{n} \wedge h\^{r}.
\end{align*}
Finally, 
$$
(\Lambda\^{2}_\alpha R)^2 \wedge \alpha\^{n+r} 
= (\Lambda_\omega \Lambda_h R)^2 \wedge \omega\^{n} \wedge h\^{r}
= (\tr_\omega c_1)^2 \wedge \omega\^{n} \wedge h\^{r},
$$
again for degree reasons and commutativity of the adjoints of the Lefschetz
operators. From this we reap
\begin{align*}
    \lvert R \rvert^2 \omega\^{n} \wedge h\^{r}
    &= 2 c_2 \wedge \omega\^{n-2} \wedge h\^{r}
    - (\Lambda_\omega R)^2 \wedge \omega\^{n} \wedge h\^{r-2} 
    \\
    &\qquad\qquad
    {}- 2 (\tr_\omega c_1)^2 \omega\^{n} \wedge h\^{r}
    - c_1^2 \wedge \omega\^{n-2} \wedge h\^{r} 
    \\
    &\qquad\qquad
    {}+ 3 (\tr_\omega c_1)^2 \wedge \omega\^{n} \wedge h\^{r}
    \\
    &= (2 c_2 - c_1^2) \wedge \omega\^{n-2} \wedge h\^{r} 
    \\
    &\qquad\qquad
    + (\tr_\omega c_1)^2 \wedge \omega\^{n} \wedge h\^{r}
    - (\Lambda_\omega R)^2 \wedge \omega\^{n} \wedge h\^{r-2}
\end{align*}
which yields
$$
    \lvert R \rvert^2 \omega\^{n}
    = (2 c_2 - c_1^2) \wedge \omega\^{n-2}
    + (\tr_\omega c_1)^2 \, \omega\^{n}
    - \Lambda\^{2}_h(\Lambda_\omega R)^2 \, \omega\^{n}.
$$
We now use the formula for the norm of a $(1,1)$--form and see that
$$
\lvert \Lambda_\omega R \rvert_h^2
= (\Lambda_h\Lambda_\omega R)^2 - \Lambda\^{2}_h(\Lambda_\omega R)^2
= (\tr_\omega c_1)^2 - \Lambda\^{2}_h(\Lambda_\omega R)^2,
$$
thus obtaining our first announced result (in equivalent notation):
$$
    \lvert R \rvert^2 \, \omega\^{n}
    = (2 c_2 - c_1^2) \wedge \omega\^{n-2}
    + \lvert \Lambda_\omega R \rvert_h^2 \, \omega\^{n}.
$$

Now assume that $(E,h)$ is Hermite--Einstein. By definition, this means
that $\tr_\omega \curv = \lambda \id_{E}$. Under our isometries, this
translates into $\Lambda_\omega R = \lambda h$. The factor $\lambda$
satisfies $r \lambda = \tr_\omega c_1$, so we get
\begin{align*}
    0 \leq 
    \lvert R \rvert^2 \, \omega\^{n}
    &= (2 c_2 - c_1^2) \wedge \omega\^{n-2}
    + r |\lambda|^2 \, \omega\^{n}.
    \\
    &= (2 c_2 - c_1^2) \wedge \omega\^{n-2}
    + \tfrac 1r (\tr_\omega c_1)^2 \omega\^{n}.
\end{align*}
Multiplying by $r$ and rearranging gives 
$$
0 \leq
r \lvert R \rvert^2 \, \omega\^{n}
= (2r c_2 - (r-1)c_1^2) \wedge \omega\^{n-2}
+ \lvert c_1 \rvert^2 \omega\^{n}.
$$
Proposition~\ref{prop:CS} below, which is just the Cauchy--Schwarz
inequality in disguise, says that
$$
\lvert c_1 \rvert^2 \leq r \lvert R \rvert^2,
$$
with equality if and only if $R = u \wedge \omega$, where $u$ is the
pullback of a form on $E$. By the Hermite--Einstein condition we
necessarily have $u = (\lambda/n) h$ in that case. This proves the
Kobayashi--L\"{u}bke inequality.
\end{proof}



\begin{prop}
\label{prop:CS}
We have $\lvert \Lambda_\omega R\rvert^2 \leq n \lvert R \rvert^2$ and
$\lvert c_1 \rvert^2 = \lvert \Lambda_h R\rvert^2 \leq r \lvert R
\rvert^2$, with equalities if and only if $R = u \wedge \omega$ or $R = v
\wedge h$, where $u$ and $v$ are pullbacks of $(1,1)$--forms from $E$ and
$V$, respectively. 
\end{prop}

\begin{proof}
    We just prove the first result since the proof of the second
differs from that in notation only.  The primitive decomposition of
$R$ as a $(2,2)$--form on $E \oplus V$ is 
$$
R = r_0 \omega \wedge h + r_1^\omega \wedge h + r_1^h \wedge \omega
+ r_2.
$$
Here $r_0$ is a scalar, $r_1^h$ is a primitive form that's a pullback from
$E$, similar for $r_1^\omega$. By orthogonality we see that $\Lambda_\omega
r_j = 0$ for these primitive forms. This gives 
$$
\Lambda_\omega R = n (r_0 h + r_1^h),
$$
so
\begin{align*}
\lvert \Lambda_\omega R \rvert^2 
&= n^2 (\lvert r_0 h\rvert^2 +\lvert r_1^h\rvert^2)
\\
&= n (\lvert r_0 \omega \wedge h\rvert^2 
+ \lvert r_1^h \wedge \omega \rvert^2)
\\
&\leq
n(\lvert r_0 \omega \wedge h \rvert^2 
+ \lvert r_1^h \wedge \omega 
+ r_1^\omega \wedge h \rvert^2 
+ \lvert r_2 \rvert^2 )
= n \lvert R \rvert^2
\end{align*}
with equality if and only if $R = u \wedge \omega$ for a form $u$ that's a pullback from $E$.
\end{proof}


\begin{exam}
Let $\omega$ be the Fubini--Study metric on $\PP^n$.  Then $c_k = (-1)^k
\binom{n+1}{k} \omega^k$ and $\Ric \omega = (n+1)\omega$. This gives
\begin{align*}
    0 \leq
    \Bigl\lvert \curv_{\omega_{\mathrm{FS}}} \Bigr\rvert^2 \omega\^n
    & = (2 c_2 - c_1^2) \wedge \omega\^{n-2} + n(n+1)^2 \omega\^{n}
    \\
    &= \Bigl(2 \tbinom{n+1}{2} - (n+1)^2\Bigr) \omega^2 \wedge \omega\^{n-2}
    + n \omega\^{n}
    \\
    &= \bigl(
    (n(n+1) - (n+1)^2) * 2 \tbinom{n}{2} + n(n+1)^2
    \bigr) \omega\^{n}
    \\    
    &= n(n+1)(-(n-1) + n + 1) \omega\^{n}
    \\
    &= 2n(n-1) \omega\^n.
\end{align*}
\end{exam}



\section{Sketches of higher Chern class inequalities}


The naive way forward is to imagine that calculating $\lvert \bigwedge^k\!
\curv_E \rvert^2$ will give an inequality involving the $k$--th Chern
class of $E$. This both works and not: The norm of the $k$--th exterior
power of the curvature form does involve the $k$--th Chern class, but also
many mixed traces $(\Lambda_\omega\^l \bigwedge^k\!\curv)^2 \wedge
\omega\^{n-2(2k-l)}$ that have no clear cohomological or geometric meaning. 

These are entirely comparable to the terms $\lvert \Lambda_\omega \curv
\rvert^2$ and $(\tr_\omega c_1)^2$ that show up in the norm of a curvature
tensor. The way out there was to assume that the vector bundle was
Hermite--Einstein, which gave cohomological meaning to the term $\lvert
\Lambda_\omega \curv \rvert^2$. Another possibility would have been to
specialize to the case of a K\"{a}hler manifold and assume $(E,h) =
(T_X,\omega)$. The additional symmetries of the curvature tensor of a
K\"{a}hler metric, as compared to an arbitrary Hermitian metric, entail
that $\Lambda_\omega \curv = \Lambda_h \curv = \Ric \omega$. Assuming
further that $(X,\omega)$ is K\"{a}hler--Einstein then gives cohomological
meaning to all the terms involved in the norm of $\curv$.

We can hope that the symmetries of the curvature tensor of a K\"{a}hler
metric ensure that $\Lambda_\omega\^l \bigwedge^k\! \curv = \Lambda_h\^l
\bigwedge^k\! \curv$ for all $l$ and $k$, like they ensure that all the
Ricci-forms of a K\"{a}hler metric agree. This would at least simplify
the terms appearing in the expansion of the norm of an exterior power of
$\curv$. It is not at all clear whether assuming $(X,\omega)$ to be
K\"{a}hler--Einstein then further simplifies those terms.

It is also probable that if this method is capable of giving inequalities
involving higher Chern classes, then considering only the terms $\lvert
\bigwedge^k\!  \curv_E \rvert^2$ is not enough to do so. If we're
interested in an inequality involving the $k$--th Chern class of $X$, then
I think we would also need to consider scalar products 
$$
\Bigl\la 
\bw{p}\! \curv_E \wedge \alpha\^{k-p}, \bw{k-p}\! \curv_E \wedge \alpha\^{p} 
\Bigr\ra,
\quad p \leq k.
$$
These would involve products $c_p c_{k-p}$ of lower Chern classes. I guess
this by analogy with the terms in the Chern character
$$
\displaylines{
\operatorname{ch}(E)
= \tr\Bigl(\exp\Bigl(\curv\Bigr)\Bigr)
= \operatorname{rank}(E)
+ c_1 + \tfrac{1}{2}(c_1^2 - 2c_2)
+ \tfrac{1}{6}(c_1^3 - 3c_1c_2 + 3c_3)
\hfill\cr\hfill
{}+ \tfrac{1}{24}(c_1^4 -4 c_2 c_1^2 + 4 c_3 c_1 + 2 c_2^2 - 4 c_4)
+ \cdots,
}
$$
who I feel have a relationship to our calculations that I can't explain
right now (there is something more than the obvious ``oh, they both
involve exterior powers'' here). If this is
the case then triple products like $c_1 c_2 c_3$ start appearing around $k
= 6$, which would further complicate things.

In the K\"{a}hler case, matters are also complicated by the need to
calculate things like $L\^{n-2(2k-p)}(\Lambda\^p L\^{q} \bw{p-q}\curv)^2$.
For this we should try to commute $\Lambda\^p$ and $L\^q$. This can be done; compare with a formula for $[L^p,\Lambda]$ in Huybrecht's~\cite[Chapter~1.2]{Huy}.

\subsection*{Bessel function}

Let 
$$
f(z) = J_0(2\sqrt z) = \sum_{m\geq0} (-1)^m \frac{1}{m!^2} z^m.
$$
This is a Bessel function of the first kind and if we set $z = xy$ and look at its reciproque we find
$$
\frac{1}{f(xy)} = \sum_{l\geq 0} (-1)^l b_l x\^l y\^l;
$$
see \cite{Carlitz,Riordan}.  We let these functions operate on the space
of sesquilinear operators on the exterior algebra $\bw{*} V^*$, by
declaring that $x\^a y\^b(\Lambda, \Lambda)(u,\overline v) := \Lambda\^a u
\wedge \Lambda\^b \overline v$. We can then write
$$
\eqalign{
(-1)^{k(k+1)/2} u \wedge \overline{\I v} \wedge \omega\^{n-p-q}
    &= 
    \pi_{n,n}\Bigl(
    f(xy)(\Lambda,\Lambda)(u,\overline{\I v})
    \wedge \exp(\omega)
    \Bigr),
    \cr
(-1)^{k(k+1)/2}
\la u,v \ra \, \omega\^n
&= 
\pi_{n,n}\Bigl(
\frac{1}{f(xy)}(\Lambda,\Lambda)(u,\overline{\I v})
\wedge \exp(\omega)
\Bigr),
}
$$
where $\pi_{n,n} : \bw{*} V^* \to \bw{n,n} V^*$ is the orthogonal
projection into the subspace of $(n,n)$--forms. This formula holds for
arbitrary $(p,q)$--forms, but proving it for an arbitrary form of
mixed bidegree should be easy.

If we get rid of the projection morphism we have
$$
\alpha(u,v)
= 
\frac{1}{f(xy)}(\Lambda,\Lambda)(u,\overline{\I v})
\wedge \exp(\omega),
$$
where $\alpha$ is some sesquilinear operator on the exterior algebra. 
Exactly which operator is probably of no importance; if we think of our
forms as living on a compact manifold then we're really interested in
the integral of the inner product over the manifold and there all forms
of bidegree other than $(n,n)$ disappear. Compare with the product of
the Chern character and Todd class in Hirzebruch--Riemann--Roch theorem,
where no one cares what the classes of degree $< 2n$ are.



\section{Variation of K\"ahler structures}


Let
$$
U = 
\Bigl\{ \alpha = \beta + i \omega \in \bw{1,1}V^* 
\Bigm\vert \Im \alpha = \omega \text{ is an inner product} 
\Bigr\}
$$
be the space of complexified inner products on $V$. Let $E^{k} \to U$
denote the trivial holomorphic vector bundle whose fiber is
$\bw{k}V^{*}$. This bundle is equipped with a Hermitian metric $h$,
defined at each point $\alpha$ by the inner product $\Im \alpha$. We would
like to calculate the curvature tensor of $h$.

We have
$$
(-1)^{k(k+1)/2}
\la u,v \ra \, \omega\^n
= 
\pi_{n,n}\Bigl(
\frac{1}{f(xy)}(\Lambda,\Lambda)(u,\overline{\I v})
\wedge \exp(\omega)
\Bigr),
$$
where $\pi_{n,n} : \bw{*} V^* \to \bw{n,n} V^*$ is the orthogonal
projection into the subspace of $(n,n)$--forms. I hope the curvature
calculations are formal and that the differential equation that the Bessel
function $f$ satisfies gives that the metric $h$ is flat: The differential
equation should let us remove the second derivatives of the metric tensor
from the curvature form, so it should depend on its first derivatives
only. Such a curvature form should be flat.

First recall that the function $f$ is defined as
$$
f(z) = J_0(2\sqrt z),
$$
where $J_0$ is a Bessel function of the first order. As such it satisfies
the differential equation
$$
z^{2} J_0'' + z J_0' + z^{2} J_0 = 0.
$$
This translates into
$$
-\tfrac 12 \sqrt z f'' + \sqrt z f' + z^{2} f = 0
$$
for the function $f$.



\subsection*{Random thought}

We think variation of K\"ahler structures is mirror to variation of Hodge
structures. Hodge structures admit a filtration. What could a
mirror filtration of K\"ahler structures be? A candidate is the filtration
of $\bw{k}V^{*}$ by primitive forms, we could even patch it together into
an increasing/decreasing filtration. These should probably not vary
holomorphically, but how? Symplectically? (Does that even make sense?)



\section{Variation of Hodge structures}


Let $\pi : X \to S$ be a family of compact K\"ahler manifolds over a
smooth base $S$. We denote by
$$
E^k = \RR^k \pi_* \ZZ \otimes_{\CC} \OO_S
$$
the holomorphic vector bundle whose fiber over a point $s$ is $E^k_s =
H^k(X_s,\CC)$. Since this vector bundle is associated to a local system it
is equipped with a flat connection $\nabla$, the Gauss--Manin connection.

Let $[\omega_s]$ be a family of K\"ahler classes on $X$, so each
$[\omega_s]$ is a K\"ahler class on $X_s$ and the classes vary smoothly
with $s$. If we so wish, we can pick a smooth family of K\"ahler metrics
$\omega_s$ so the cohomology class of $\omega_s$ is $[\omega_s]$. We then
get a Hermitian inner product $h$ on $E^{k}$ defined either by using
harmonic representatives of $k$--forms or by defining the inner product
directly on the level of cohomology through $[\omega_s]$.

Recall that the family $(X,[\omega])$ is \emph{polarized} if $\nabla
[\omega] = 0$.


\begin{prop}
    If the family $(X,[\omega])$ is polarized then the Gauss--Manin
connection is the Chern connection of $h$.
\end{prop}


\begin{proof}
Let $u$ be a local section of $E^{k}$. We write $u = \sum L^{k-j}u_j$ for
the fiberwise primitive decomposition of $u$ with respect to $[\omega]$. 
Since each $u_j$ is a primitive $j$--class on each manifold in the family
we have $L\^{n-j-1} u_j = 0$. This gives
$$
0 = \nabla(L\^{n-j-1} u_j) =  L\^{n-j-1} \nabla u_j
$$
because $\nabla$ and $L$ commute since the family is polarized, so
$\nabla_\xi u_j$ is again a primitive $j$--class for any tangent field
$\xi$ of $S$. Applying Weil's formula~\ref{16} now shows that 
$\nabla\!\! *\! u = * \nabla u$. We can now conclude either by using 
$$
\langle u, v \rangle = \int_X  u \wedge * \overline v,
$$
or by noting that the above leads to $\nabla \Lambda = \Lambda \nabla$,
so our formula for the inner product shows that $\nabla$ is compatible
with $h$.
\end{proof}


A minor application is the seminegativity of all direct images
$\RR^{p}\pi_* \OO_{X/S}$, since they are holomorphic subbundles of $E^k$
by Griffiths. Slightly more generally the Hodge filtration of $E^{k}$ is
composed of seminegative bundles. I wonder if Griffiths already proved
that? (Griffiths at least proved that the quotient bundles of the Hodge
filtration are seminegative, I don't think he mentioned the Hodge
filtration itself. He used the Hodge--Riemann intersection form for this,
not the inner product on cohomology.)


\bibliographystyle{amsalpha}
\bibliography{cci}

\end{document}
